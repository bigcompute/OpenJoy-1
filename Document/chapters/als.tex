\section{Emitted Particle Angular Distribution}
There are many reactions from which particles are emitted. Refer to Table \ref{tab:mt-meaning} for the meaning of the reactions. The direction of flight of the emitted particle is described by the angular distributions. The scattered particles are usually to be azimuthal symmetric, which means that for incoming direction $\hat{\Omega}$ (a unit vector) and outgoing direction $\hat{\Omega}'$ (a unit vector). The probability of emitting depends on only on the cosine of the angular between $\hat{\Omega}$ and $\hat{\Omega}'$, which is defined as:
\begin{eqnarray}
\mu&=&\hat{\Omega}\cdot\hat{\Omega}',\;-1\leq\mu\leq1.
\end{eqnarray}
The distribution function $f(\mu)$ is normalized with the condition:
\begin{eqnarray}
\int_{-1}^{1}f(\mu)\,d\mu &=& 1.
\label{eq:als_norm}
\end{eqnarray}
The angular distribution in general depends on the energy of outgoing particle as well. Even, the energy and the angular cosine may couple together. In this chapter, the discussion treats solely the angular distribution. In this chapter, we discuss the approaches to represent the angular distribution, and the linearization approach to make it suitable for computation. First we discuss the representation formats in the ENDF data files. Then, we discuss the algorithm to convert them into computation friendly format.

\section{Representation in ENDF Data File}
In ENDF data file, the angular distribution is represented in either a Legendre polynomial expansion or the ENDF interpolation table. 
\subsection{Legendre Expansion}
For functions $f(\mu)$ defined in the interval $[-1,1]$, there is an orthogonal expansion (in the sense of mathematics) can be used to approximate the function:
\begin{eqnarray}
f(\mu) &\approx& \sum_{l=0}^{N}\,a_l\,\frac{2l+1}{2}\,P_l(\mu),
\end{eqnarray}
where $P_l(\mu)$ is the $l$th order Legendre polynomial. There is an recursive definition for $P_l(\mu)$:
\begin{eqnarray}
P_0(\mu) &=& 1,\\
P_1(\mu) &=& \mu,\\
P_l(\mu) &=& \frac{2l-1}{l}\mu P_{l-1}(\mu)-\frac{l-1}{l}P_{l-2}(\mu),\;l\geq 2.
\end{eqnarray}
The coefficients $a_l$ can be calculated by the formula:
\begin{eqnarray}
a_l &=& \int_{-1}^{-1}f(\mu)P_l(\mu)\,d\mu.
\end{eqnarray}
Apply the normalization \ref{eq:als_norm}, the Legendre coefficients are assumed to satisfy:
\begin{eqnarray}
a_0 &=& \frac{1}{2}.
\end{eqnarray}
In the ENDF data file, the Legendre coefficients $a_1,a_2,\dots a_N$ are given.

\subsection{ENDF Interpolation Laws}
As discussed in previous chapter, an ENDF interpolation function can provided to describe the angular distribution functions.

\section{Linearization of Angular Distribution}
To make the angular distribution suitable for computation, the function needs to be linearized. The linearization used in OpenJOY follows the `Inverted Stack' algorithm in previous discussions of cross section linearization. At the beginning, two function points evaluated at $\mu=\pm1$ are inserted in the initial stack at the beginning of the algorithm.
