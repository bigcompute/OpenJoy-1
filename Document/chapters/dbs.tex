Doppler broadening is an important physical phenomena need to take consideration for neutron transport calculation to get the collision probability correct. We first discuss the mathematical background of this issue, and then give the theory behind the widely used SIGMA1 method.
\section{Mathematical Background of Doppler Broadening}
Let $\vec{v}$ be the velocity of the neutron, let $\vec{V}$ be the velocity of the target nucleus, then we can define an averaged collision cross section for target nuclide moving with a probability $f(\vec{V})$.
\begin{eqnarray}
\int_{R^3}f(\vec{V})d\vec{V}&=&1,
\end{eqnarray}
\noindent where if $\vec{V}=(x,y,z)$ or $\vec{V}=(r\cos\theta\sin\phi,r\sin\theta\sin\phi,r\cos\phi)$, for a general function $K(\vec{V})=K(x,y,z)$:
\begin{eqnarray}
\int_{R^3}K(\vec{V})d\vec{V}&=&\int_{0}^\infty\int_{0}^\infty\int_{0}^\infty K(\vec{V})dxdydz,\\
\int_{R^3}K(\vec{V})d\vec{V}&=&\int_{0}^\infty\int_{0}^{\pi}\int_{0}^{2\pi} K(\vec{V}) r^2\sin\phi d\theta d\phi dr,
\end{eqnarray}
\noindent the averaged cross section is:
\begin{eqnarray}
\bar{\sigma}(\vec{v})&=&\frac{1}{|\vec{v}|}\int_{R^3}|\vec{v}-\vec{V}|\sigma(|\vec{v}-\vec{V}|)f(\vec{V})d\vec{V}.
\end{eqnarray}
\noindent In the thermal motion case, the distribution function $f$ follows the free-gas thermal motion law, which at temperature $T$ is:
\begin{eqnarray}
f(\vec{V})&=&f^{*}(|\vec{V}|, T)=(\frac{M}{2\pi k T})^{3/2}e^{-M|\vec{V}|^2/2kT},
\end{eqnarray}
\noindent which depends only on the magnitude of $\vec{V}$. In this case we are able to define a cross section only depends on the magnitude of neutron velocity $\vec{v}$:
\begin{eqnarray}
\bar{\sigma}(|\vec{v}|,T)&=&\frac{1}{|\vec{v}|}\int_{R^3}|\vec{v}-\vec{V}|\sigma(|\vec{v}-\vec{V}|)f^{*}(|\vec{V}|, T)d\vec{V}.
\end{eqnarray}
\noindent If we define $v_r=|\vec{v}-\vec{V}|$, and denote $v$ and $V$ as the speed of $\vec{v}$ and $\vec{V}$:
\begin{eqnarray}
v&=&|\vec{v}|,\\
V&=&|\vec{V}|.
\end{eqnarray}
\noindent It is possible to rewrite the integral in scalar quantities as:
\begin{eqnarray}
\bar{\sigma}(v,T)&=&\frac{2\pi}{v^2}\int_0^{\infty}\int_{|v-V|}^{v+V}v_r^2\sigma(v_r)Vf^{*}(V,T)dv_rdV.
\end{eqnarray}
\noindent Change the order of integration, we get:
\begin{eqnarray}
\bar{\sigma}(v,T)&=&\frac{2\pi}{v^2}\int_0^{\infty}\int_{|v-v_r|}^{v+v_r}v_r^2\sigma(v_r)Vf^{*}(V,T)dVdv_r.
\end{eqnarray}
\noindent If we define the thermal speed $v_{th}=\sqrt{kT/M}$, $\omega=v/\sqrt{2}v_{th}$, and $\omega_r=v_r/\sqrt{2}v_{th}$:
\begin{eqnarray}
\bar{\sigma}(\omega,T)&=&\frac{1}{\omega^2\sqrt{\pi}}\int_0^{\infty}\omega_r^2\sigma(\omega_r)\left(e^{-(\omega-\omega_r)^2}-e^{-(\omega+\omega_r)^2}\right)d\omega_r.
\end{eqnarray}
\noindent In terms of the more frequently used energy quantities of neutron $E=\frac{1}{2}mv^2$, where $m$ is the neutron mass. We define the additional quantities:
\begin{eqnarray}
\alpha &=& \frac{M}{2kT},\\
x &=& \sqrt{\alpha}\,v = \sqrt{\alpha \frac{2E}{m}}=\sqrt{\frac{ME}{mkT}},
\end{eqnarray}
where $x$ is dimensionless. We define $\sigma(x,T)$ as the cross section as a function of the variable $x$ and temperature $T$, the formula is rewrite as:
\begin{eqnarray}
x^2\sigma(x,T) &=& \frac{1}{\sqrt{\pi}}\int_{0}^{\infty}x_r^2\sigma(x_r,0)\left(e^{-(x-x_r)^2}-e^{-(x+x_r)^2}\right)dx_r,
\end{eqnarray}
in which we have $x=\omega$ and $x_r=\omega_r$. In a more general case, if we start at temperature $T_1$ and want to get doppler broadened cross section at temperature $T_2$, and define $\alpha=M/2k(T_2-T_1)$, and $x=\sqrt{ME/mk(T_2-T_1)}$, we have:
\begin{eqnarray}
x^2\sigma(x,T_2) &=& \frac{1}{\sqrt{\pi}}\int_{0}^{\infty}x_r^2\sigma(x_r,T_1)\left(e^{-(x-x_r)^2}-e^{-(x+x_r)^2}\right)dx_r.
\end{eqnarray}
For simplicity, if we change the variable $x_r,x$ by $x,y$, we have:
\begin{eqnarray}
y^2\sigma(y,T_2) &=& \frac{1}{\sqrt{\pi}}\int_{0}^{\infty}x^2\sigma(x,T_1)\left(e^{-(y-x)^2}-e^{-(y+x)^2}\right)dx,
\end{eqnarray}
which is the formula we usually see in the textbooks.

\section{Theory of SIGMA1 Method}
The SIGMA1 method is probably the most widely used algorithm in the doppler broadening codes. It starts from a linearly interpolated cross section data points, which appears in the PENDF format in the NJOY system. We first define this linearly interpolated data points and derive the algorithm on it.
\subsection{Grid for Data Points}
Given the `cold'  cross section on a $x$ grid $(x_n,\sigma_n),n=1\dots N$ with additional data points:
\begin{eqnarray}
x_0 &=& 0,\\
x_{N+1} &=& \infty.
\end{eqnarray}
Let the cross section have a $1/x$ interpolation below the given data points, i.e. in the interval $(0,x_1]$/ The cross section is constant above the given data points, i.e. in the interval $(x_N,\infty)$. Therefore, the cross sections as a function of $x$ at the cold temperature $T_1$ are given by:
\begin{eqnarray}
\sigma(x,T_1)&=&\left\{
\begin{array}{r @{\;:\;} l}
\frac{x_1\sigma_1}{x} & 0<x\leq x_1\\
\sigma_n+s_n(x^2-x_n^2) & x_n<x\leq x_{n+1},\;\;1\leq n\leq {N-1}\\
\sigma_N & x_N<x<\infty
\end{array}\right.,
\label{eq:dbxsec}
\end{eqnarray}
where the slope is given by:
\begin{eqnarray}
s_n &=& \frac{\sigma_{n+1}-\sigma_n}{x_{n+1}^2-x_n^2},\;\;1\leq n\leq {N-1}.
\end{eqnarray}
\subsection{SIGMA1 Algorithm}
The doppler broadened cross section at `hot' temperature $T_2$ is given by the formula:
\begin{eqnarray}
\sigma(y,T_2) &=& \sigma^{*}(y,T_2) - \sigma^{*}(-y,T_2),\\[1ex]
\sigma^{*}(y,T_2) &=& \frac{1}{\sqrt{\pi}\,y^2}\int_{0}^{\infty}x^2\,\sigma(x,T_1)\,e^{-(x-y)^2}dx.
\end{eqnarray}
Plug in the formula Equation (\ref{eq:dbxsec}) for interpolated `cold' cross section formula:
\begin{eqnarray}
\sigma^{*}(y,T_2) &=& x_1\sigma_1\frac{1}{\sqrt{\pi}\,y^2}\int_{0}^{x_1}x\,e^{-(x-y)^2}dx+\nonumber\\
&&\frac{1}{\sqrt{\pi}\,y^2}\sum_{n=1}^{N-1}\int_{x_{n}}^{x_{n+1}}x^2\,[\sigma_n+s_n(x^2-x_n^2)]\,e^{-(x-y)^2}dx+\nonumber\\
&&\sigma_N\frac{1}{\sqrt{\pi}\,y^2}\int_{x_N}^{\infty}x^2\,e^{-(x-y)^2}dx
\end{eqnarray}
Next, we define the $H$ function for simplifying the doppler broadening formula:
\begin{eqnarray}
H_m(a,b) &=& \frac{1}{\sqrt{\pi}}\int_{a}^{b}z^me^{-z^2}\;dz,
\end{eqnarray}
with the eqivalent representation:
\begin{eqnarray}
H_m(a,b) &=& F_m(a)-F_m(b),\\[1ex]
F_m(a) &=& \frac{1}{\sqrt{\pi}}\int_{a}^{\infty}z^me^{-z^2}dz.
\end{eqnarray}
The Taylor expansions for the case that $a$ close to $b$ are:
\begin{eqnarray}
H_m(a,b) &=& F_m(a)-F_m(b),\\[1ex]
&=& -F'_m(a)(a-b)+\mathcal{O}((a-b)^2),\\[1ex]
&=& \frac{1}{\sqrt{\pi}}a^me^{-z^2}(a-b)+\mathcal{O}((a-b)^2).
\end{eqnarray}
The recursive formula for the $H$ function can be written as:
\begin{eqnarray}
F_0(a) &=& \frac{1}{2}\,\mbox{erfc}(a),\\
F_1(a) &=& \frac{1}{2\sqrt{\pi}}\,e^{-a^2},\\
F_m(a) &=& \frac{n-1}{2}F_{m-2}(a)+a^{m-1}F_1(a),\,m\geq 2,
\end{eqnarray}
and the formula for lower order $F_n$ are:
\begin{eqnarray}
F_2(a) &=& \frac{1}{4}\,\mbox{erfc}(a)+\frac{1}{2\sqrt{\pi}}\,a\,e^{-a^2},\\
F_3(a) &=& \frac{1}{2\sqrt{\pi}}\,\left(1+a^2\right)\,e^{-a^2},\\
F_4(a) &=& \frac{3}{8}\,\mbox{erfc}(a)+\frac{1}{2\sqrt{\pi}}\left(\frac{3}{2}a+a^3\right)\,e^{-a^2}.
\end{eqnarray}
As a result of this, formula for $\sigma^{*}$ is simplified as:
\begin{eqnarray}
\sigma^{*}(y,T_2) = \sum_{n=1}^{N-1}\left[A_{n}(y)(\sigma_n-s_nx_n^2)+B_{i}(y)s_n\right]+x_1\sigma_1\left[C(y)\right]+\sigma_N\left[B_N(y)\right],
\end{eqnarray}
where we define $A_{i}(y)$ and $B_{i}(y)$ as:
\begin{eqnarray}
H_{m,n}(y) &=& H_m(x_n-y, x_{n+1}-y),\\[1ex]
A_{n}(y) &=& \frac{1}{y^2}H_{2,n}(y)+\frac{2}{y}H_{1,n}(y)+H_{0,n}(y),\\
B_{n}(y) &=& \frac{1}{y^2}H_{4,n}(y)+\frac{4}{y}H_{3,n}(y)+6H_{2,n}(y)+4yH_{1,n}(y)+y^2H_{0,n}(y),\\
C(y) &=& \frac{1}{y^2}H_{1,0}(y)+\frac{1}{y}H_{0,0}(y).
\end{eqnarray}
There is a special case for $n=N$:
\begin{eqnarray}
H_{m,N}(y) &=& H_m(x_N-y,\infty),\nonumber\\
&=& F_m(x_N-y) - F_m(\infty),\nonumber\\
&=& F_m(x_N-y) - \frac{1}{\sqrt{\pi}}\int_{\infty}^{\infty}z^me^{-z^2}dz,\nonumber\\
&=& F_m(x_N-y).
\end{eqnarray}

\section{SIGMA1 Code Approximation}
In earlier computer implementation, for $y<64$, there is an additional approximation that used. The key is to assume that:
\begin{eqnarray}
\int_{x_n}^{x_{n+1}}x^2\left[\sigma_n+s_n(x^2-x_n^2)\right]e^{-(x-y)^2}dx \approx 
\int_{x_n}^{x_{n+1}}x^2\,\frac{\sigma_n+\sigma_{n+1}}{2}e^{-(x-y)^2}dx.
\end{eqnarray}
With this approximation,
\begin{eqnarray}
\sigma^{*}(y,T_2) &\approx& \sum_{n=1}^{N-1}\frac{\sigma_n+\sigma_{n+1}}{2}\left[A_{n}(y)\right]+x_1\sigma_1\left[C(y)\right]+\sigma_N\left[B_N(y)\right].
\end{eqnarray}
No justification is provided here. In the OpenJOY code, the version with no approximations is used.

%
%\subsection{BROADL}
%The BROADL subroutine is responsible for doppler broadening the cross section. For the case that $x_n>y$, the integral during the $[x_n,x_{n+1})$ portion is given by the following formula for $I_n(y)$:
%\begin{eqnarray}
%I_n(y) &=& I_n^{*}(y) - I_n^{*}(-y),\\[1ex]
%I_n^{*}(y) &=& (\sigma_{n}+\sigma_{n+1})\cdot(f2a_n(y)-f2b_n(y)),\\[1ex]
%f2a_n(y) &=& (y^2+\frac{1}{2})\,\mbox{erfc}(x_n-y)+\frac{1}{\sqrt{\pi}}(x_n+y)e^{-(x_n-y)^2},\\[1ex]
%f2b_n(y) &=& (y^2+\frac{1}{2})\,\mbox{erfc}(x_{n+1}-y)+\frac{1}{\sqrt{\pi}}(x_{n+1}+y)e^{-(x_{n+1}-y)^2}.
%\end{eqnarray}
%When $y=x_n$, we have the simplification:
%\begin{eqnarray}
%f2a_n(y) &=& \left(y^2+\frac{1}{2}\right) + \frac{2}{\sqrt{\pi}}y.
%\end{eqnarray}
%After apply the previous definitions, we have:
%\begin{eqnarray}
%\frac{1}{4y^2}I_n(y) &=& \frac{\sigma_n+\sigma_{n+1}}{2}A_n(y).
%\end{eqnarray}
%The key approximation behind is:
%\begin{eqnarray}
%\int_{x_n}^{x_{n+1}}x^2\left[\sigma_n+s_n(x^2-x_n^2)\right]e^{-(x-y)^2}dx \approx 
%\int_{x_n}^{x_{n+1}}x^2\,\frac{\sigma_n+\sigma_{n+1}}{2}e^{-(x-y)^2}dx.
%\end{eqnarray}
%
%\subsection{BROADH}
%The BROADH subroutine provides the doppler broadening algorithm for the energy region higher than the one applicable for BROADL, the difference is that the approximation is not used. I.e. the direct summed term for the energy between $x_n$ and $x_{n+1}$ for the case $y>x_n$ is:
%\begin{eqnarray}
%A_n(y)\sigma_n+(B_n(y)-A_n(y)x_n^2)s_n,
%\end{eqnarray}
%which is evaluated by the integral:
%\begin{eqnarray}
%I_n(y) &=& I_n^{*}(y) - I_n^{*}(-y),\\[1ex]
%I_n^{*}(y) &=& \sigma_{n}\cdot(f2a_n(y)-f2b_n(y))+s_n\cdot(f4a_n(y)-f4b_n(y)),\\[1ex]
%f2a_n(y) &=& \left(y^2+\frac{1}{2}\right)\,\mbox{erfc}(x_n-y)+\frac{1}{\sqrt{\pi}}(x_n+y)e^{-(x_n-y)^2},\\[1ex]
%f2b_n(y) &=& \left(y^2+\frac{1}{2}\right)\,\mbox{erfc}(x_{n+1}-y)+\frac{1}{\sqrt{\pi}}(x_{n+1}+y)e^{-(x_{n+1}-y)^2},\\[1ex]
%f4a_n(y) &=& \left(\left(y^2+\frac{1}{2}\right)(y^2-x_n^2)+\frac{5}{2}y^2+\frac{3}{4}\right)\,\mbox{erfc}(x_n-y)+\nonumber\\[1ex]
%&& \frac{1}{\sqrt{\pi}}\left((x_n+y)\left(x_n^2+\frac{3}{2}+y^2-x_n^2\right)+y\right)e^{-(x_{n}-y)^2},\\[1ex]
%f4b_n(y) &=& \left(\left(y^2+\frac{1}{2}\right)(y^2-x_n^2)+\frac{5}{2}y^2+\frac{3}{4}\right)\,\mbox{erfc}(x_{n+1}-y)+\nonumber\\[1ex]
%&& \frac{1}{\sqrt{\pi}}\left((x_{n+1}+y)\left(x_{n+1}^2+\frac{3}{2}+y^2-x_n^2\right)+y\right)e^{-(x_{n+1}-y)^2}.
%\end{eqnarray}
%When $y=x_n$, we have the simplification:
%\begin{eqnarray}
%f2a_n(y) &=& \left(y^2+\frac{1}{2}\right)+\frac{2}{\sqrt{\pi}} y,\\
%f4a_n(y) &=& \left(\frac{5}{2}y^2+\frac{3}{4}\right) + \frac{2}{\sqrt{\pi}} y (y^2+2) .
%\end{eqnarray}
%The formula is given by:
%\begin{eqnarray}
%\frac{1}{2y^2}I_n(y).
%\end{eqnarray}
%After reverse reduction, the term $\frac{1}{2y^2}I_n(y)$ is reconstructed as:
%\begin{eqnarray}
%&&\sigma_n\left((1+\frac{1}{2y^2})H_{0,n}(y)+\frac{1}{y^2}H_{2,n}(y)-\frac{1}{2y^2}H_{0,n}(y)+\frac{2}{y}H_{1,n}(y)\right)+\nonumber\\
%&&s_n\left( \left((1+\frac{1}{2y^2})(y^2-x_n^2)+\frac{5}{2}+\frac{3}{4y^2}\right)H_{0,n}(y) +\right.\nonumber\\
%&&\left. (1+\frac{3}{2y^2})(H_{2,n}(y)-\frac{1}{2}H_{0,n}(y)) + \left((1+\frac{3}{2y^2})(2y)+\frac{1}{y}\right)H_{1,n}(y) \right) \nonumber\\[1ex]
%&=& \sigma_n \left(H_{0,n}(y)+\frac{2}{y}H_{1,n}(y)+\frac{1}{y^2}H_{2,n}(y)\right)+\nonumber\\
%&& s_n\left(\left((1+\frac{1}{2y^2})(y^2-x_n^2)+2\right)H_{0,n}(y)+(1+\frac{3}{2y^2})H_{2,n}(y)+\right.\nonumber\\
%&& \left.\left((1+\frac{3}{2y^2})(2y)+\frac{1}{y}\right)H_{1,n}(y) \right).
%\end{eqnarray} 
%
%
%\subsection{BROADS}
%The BROADS subroutine provides the doppler broadening algorithm for very high energy. At this level, some more approximations can be applied. For the case that $x_n>y$, the integral during the $[x_n,x_{n+1})$ portion is given by the following formula for $I_n(y)$:
%\begin{eqnarray}
%I_n(y) &=& I^{*}_n(y) - I^{*}_n(-y),\\[1ex]
%I^{*}_n(y) &=& \sigma_n(\mbox{erfc}(x_n-y) - \mbox{erfc}(x_{n+1}-y))+s_n(f4a_n(y)-f4b_n(y)),\\[1ex]
%f2a_n(y) &=& \frac{1}{\sqrt{\pi}}(x_n+y)\,e^{-(x_n-y)^2},\\[1ex]
%f4a_n(y) &=& \left((y^2-x_n^2)+\frac{5}{2}\right)\,\mbox{erfc}(x_n-y)+f2a_n(y),\\[1ex]
%f2b_n(y) &=& \frac{1}{\sqrt{\pi}}(x_{n+1}+y)\,e^{-(x_{n+1}-y)^2},\\[1ex]
%f4b_n(y) &=& \left((y^2-x_{n}^2)+\frac{5}{2}\right)\,\mbox{erfc}(x_{n+1}-y)+f2b_n(y).
%\end{eqnarray}
%When $y=x_n$, we have the simplification:
%\begin{eqnarray}
%f2a_n(y) &=& \frac{2}{\sqrt{\pi}} y,\\
%f4a_n(y) &=& \frac{5}{2} + \frac{2}{\sqrt{\pi}} y.
%\end{eqnarray}
%The summed integral is given by:
%\begin{eqnarray}
%\frac{1}{2}I_n(y).
%\end{eqnarray}
%The key approximation is:
%\begin{eqnarray}
%\frac{x}{y} &\approx& 1,\\[1ex]
%\frac{1}{y^2}\int_{x_n}^{x_{n+1}}x^2\left[\sigma_n+s_n(x^2-x_n^2)\right]e^{-(x-y)^2}dx &\approx&\\[1ex]
%\int_{x_n}^{x_{n+1}}\left[\sigma_n+s_n(x^2-x_n^2)\right]e^{-(x-y)^2}dx&&.
%\end{eqnarray}
%The kernel approximation is given by:
%\begin{eqnarray}
%\epsilon &=& \left(\frac{x}{y}\right)^2-1,\\
%\delta &=& \left(\frac{x_n}{y}\right)^2-1,
%\end{eqnarray}
%and we can see that $\epsilon$ and $\delta$ are in the nearly same order,
%\begin{eqnarray}
%\frac{1}{y^2}\left[\left(\sigma_n-s_nx_n^2\right)x^2+s_nx^4\right] \nonumber
%\end{eqnarray}
%\begin{eqnarray}
%&=& (\sigma_n-s_ny^2(1+\delta))(1+\epsilon)+s_ny^2(1+\epsilon)^2,\nonumber\\
%&=&  \sigma_n(1+\epsilon)-s_ny^2(1+\delta+\epsilon)+s_ny^2(1+2\epsilon)+\mathcal{O}(\epsilon^2)+\mathcal{O}(\delta^2)+\mathcal{O}(\epsilon\delta),\nonumber\\
%&=& \sigma_n(1+\epsilon)+s_ny^2(\epsilon-\delta)+\mathcal{O}(\epsilon^2)+\mathcal{O}(\delta^2)+\mathcal{O}(\epsilon\delta),\nonumber\\
%&=&\sigma_n \frac{x^2}{y^2} + s_n(x^2-x_n^2) + \mathcal{O}(\epsilon^2)+\mathcal{O}(\delta^2)+\mathcal{O}(\epsilon\delta),\nonumber\\
%&=& -s_nx_n^2 + x^2\left(\frac{1}{y^2}\sigma_n+s_n\right) + \mathcal{O}(\epsilon^2)+\mathcal{O}(\delta^2)+\mathcal{O}(\epsilon\delta),\nonumber\\
%\end{eqnarray}
%\begin{eqnarray}
%-s_nx_n^2H_{0,n}(y)+\left(\frac{1}{y^2}\sigma_n+s_n\right)(y^2 H_{0,n}(y) + 2yH_{1,n}(y) + H_{2,n}(y))
%\end{eqnarray}
%As results, the summed term has the approximation:
%\begin{eqnarray}
%A_n(y)\,\sigma_n+(B_n(y)-A_n(y)x_n^2)\, s_n &\approx& E_n(y)\,\sigma_n+(F_n(y)-E_n(y)x_n^2)\, s_n
%\end{eqnarray}
%where,
%\begin{eqnarray}
%E_n(y) &=& H_{0,n}(y), \\
%F_n(y) &=& y^2 H_{0,n}(y) + 2yH_{1,n}(y) + H_{2,n}(y).
%\end{eqnarray}
%After reversely reduction, the integral terms $\frac{1}{2}I_n(y)$ are reconstructed by:
%\begin{eqnarray}
%&&\sigma_n H_{0,n}(y) + \nonumber\\
%&&s_n \left((y^2-x_n^2)H_{0,n}(y)+\frac{5}{2}H_{0,n}(y)+2yH_{1,n}(y)+H_{2,n}(y)-\frac{1}{2}H_{0,n}(y)\right)\\
%&=&\sigma_n H_{0,n}(y) + s_n \left((y^2-x_n^2)H_{0,n}(y)+2H_{0,n}(y)+2yH_{1,n}(y)+H_{2,n}(y)\right).\;\;
%\end{eqnarray}

\section{C++ Data Structure \& Application Interfaces}
\subsection{Data Structure}
The operation of doppler broadening is applied pairs of energy and cross section. A cross section pair is defined in the following data structure:
\begin{verbatim} 
struct ENDFXsecPair {
    double energyEv  = 0.;
    double sigmaBarn = 0.;
};
\end{verbatim}
To avoid potential confusion in the units of energies, we explicitly mark the unit at the end of variable names.

\subsection{Application Interfaces (APIs)}
The C++ class DBS has a few static methods (currently only one) that take a vector (C++ standard library) of pairs of energy and cross section, the mass ratio of target nuclide to neutron, and a {\em positive} temperature difference between the broadened (`hot') cross section and the un-broadened (`cold') cross section. The vector of pairs of energy and cross section is taken by reference.
\begin{program}[!htb]
\centering
\begin{verbatim} 
class DBS {
public:
    static void proceedWithSigma1
    (std::vector<ENDFXsecPair>& xspairs, 
     double massRatio, double tempK);
};
\end{verbatim}
\caption{ \label{program:ndls_cpp_api}
C++ public APIs for DBS module}
\end{program}
All methods will modify the cross section in-place, i.e. the member `sigmaBarn' in the elements of `xspairs' will be modified to the doppler broadened cross sections.

\subsubsection{Method: proceedWithSigma1}
The `proceedWithSigma1' method will doppler broaden the cross section with the SIGMA1 method as discussed in previous sections.
